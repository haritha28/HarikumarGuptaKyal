The implementation of the TrackMe system will be done service by service followed with the integration of all the services. The order in which it is be carried out depends on a number of factors like the complexity of the modules and services, the dependence of other modules on the component being implemented and to the system as a whole, and it should also take into account the possibility of discovering flaws with the proposed design. The later should be dealt in a way that, if such an unfortunate event does happen,the flaws should be found and corrected as soon as possible, to limit the cost of the change of design. 
In this sense, the components of the TrackMe, could be grouped in the following way, with the order specifying the order of implementation: \newline
1. Data4Help Service\newline 2. Automated SOS Service and Ambulance Application\newline 3. Track4Run Service\newline 4. TrackMe (Connecting the above three services)\newline

The first service to be built is Data4Help Service as the further services exploits the features of Data4Help Service. Data4Help is a base service which all the individuals and third party will have by default when they register in the application. Hence, it has to be built first as a separate service. This service does the main function of taking data from user and storing it in database and hence does not require any payment interface with might be required by other services, hence it is implemented separately.\newline 

After the first service AutomatedSOS and TrackMe Ambulance Application is implemented which is a service used by only Individuals. This is a service for the elderly individuals whose vitals are monitored and a nearby ambulance is sent whenever emergency situations are encountered. This service exploits the features of the above service and hence it is implemented after that. The details of the individuals with vitals below threshold is sent to the Ambulance Driver. The ambulance drivers have explicit applications in order to ensure fast delivery of individual's data through push notification.Hence the application must be implemented together with this service layer as there is an exchange of information from application to this service layer and vice versa.\newline

The next service implemented is Track4Run service which is basically an event which is organized by Organizers who are basically third party and individuals can perform as athlete. This service is not related to the second service by any means and hence both the services can be implemented parallely and separately and this can be implemented even before AutomatedSOS. The order of these two service wont matter but it must be implemented after Data4Help Service as it exploits the feature of that service as well.\newline

After implementing all the three services of the application, we can merge them and implement the entire working of the application connecting all the three services and how the user upgrades from Data4Help to other services and uses them. Hence this layer is implemented at the end after implementing all the services separately.