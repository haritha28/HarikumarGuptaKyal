\subsection{User Interfaces}


\subsection{Hardware Interfaces}


\subsection{Software Interfaces}
\begin{enumerate}
\item \textbf{Data4Help API}
\newline \qquad TrackMe extracts location and health of the individual from the API of the   wearable device and stores it in a database. An API of the database is created which is provided to the other services of TrackMe. This information is used for many purposes. “AutomatedSOS” and “Track4Run” uses the API of the “Data4Help” in order to provide their service. Both the services exploit the services of “Data4Help”. “AutomatedSOS” uses the API to monitor the health status of elderly people and help them by providing an ambulance. “Track4Run” uses the API to extract the location and health of individuals registered as athletes. The location will be displayed to all the users and the health status of athletes will be monitored by the third party organizers.
\item \textbf{Ambulance Application}
\newline\qquad TrackMe develops an application for all the ambulance drivers associated with major Hospitals in the city of Milan. The application fetches the details and location of all the ambulance driver. When the health status of an subscribed individual is below a threshold value, the software sends a push notification to the nearby ambulance drivers phone with the details and location of the individual. Once an ambulance driver accept the request, the software removes the alert message from the application. The driver can navigate to the location of the individual in need of emergency using the map provided in the application.
\item \textbf{City Maps}
\newline\qquad We will be using Google Maps API to facilitate the location monitoring for third party services and insertion of race location in Track4Run and race visualization.
\item \textbf{Communication Interfaces}
\end{enumerate}