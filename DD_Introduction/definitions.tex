\subsection{Definitions}
\begin{enumerate}
\item Services: Functionalities offered by the application
\item Validation: Verifying if requests provided by the 3rd party are genuine or not. If it is genuine it is accepted else rejected.
\item Upgrade: Enhancing the functionalities. 
\item Update: Adding new information or changing the previous one.
\item Threshold: A limit. Threshold on age has been kept for Automated SOS service.
\item Push Notification: The message that pops up on the Mobile. 
\end{enumerate}

\subsection{Acronyms}
\begin{enumerate}
\item DD: Design Document
\item API: Application Program Interface
\item DBMS: Data Base Management System
\item SOS: Save our Souls
\item GPS: Global Positioning System
\item GUI: Graphical User Interface
\item RASD: Requirement Analysis and Specification Document
\end{enumerate}


